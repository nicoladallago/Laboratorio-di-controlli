\documentclass[crop]{standalone}
\usepackage[italian]{babel} 
\usepackage[T1]{fontenc}
\usepackage[utf8x]{inputenc}
\usepackage{float}
\usepackage{graphicx}
\usepackage{placeins}
\usepackage{wrapfig}
\usepackage{enumitem}
\usepackage{subfig}
\usepackage{sidecap}
\usepackage{newclude}
\usepackage{amsmath}
\usepackage{amssymb}
\usepackage{epstopdf}
\usepackage{fancyhdr}
\usepackage{booktabs,array}
\usepackage[output-decimal-marker={,}]{siunitx}
\usepackage{color}
\usepackage{empheq}
\usepackage{listings}
\usepackage{babel}
\usepackage{graphicx}
\usepackage{hyperref}
\usepackage{hf-tikz}
\usepackage{tikz}
\usetikzlibrary{spy}
\usepackage{bodegraph}
\usepackage{schemabloc}
\usetikzlibrary{circuits}
\usepackage{tabularx}
\usepackage{siunitx}
\usepackage{pgfplots, pgfplotstable}
\usepackage[siunitx]{circuitikz}
\usetikzlibrary{decorations.pathreplacing}

\newcommand{\trippleSpacing}{\phantom{aaa}}	% spazio di tre caratteri
\newcommand{\singleSpacing}{\phantom{a}}	% spazio di un carattere

\begin{document}

		% font e linea grafici MATLAB
		\pgfplotsset{every axis/.append style={
			font=\large,
			line width=1pt,
			tick style={line width=0.8pt}}}
		
		\begin{figure}[H]
			\centering
			\begin{tikzpicture}[scale=0.99]
					%\draw[help lines,step=.2] (0,0) grid (4,4);
					\begin{axis}[
					grid=major,
			  		xmin=950, xmax=1300, 
			  		ymin=-10, ymax=150,     
			  		xlabel={$t \singleSpacing [s]$},
			  		ylabel={$y \singleSpacing [gradi]$},            
			  		title={Uscite del motore per $d=0.2$}, 
			  		xtick={1000,1100,...,1300},
			  		xticklabels={$1$, $1.1$, $1.2$, $1.3$},
			  		ytick={0, 100, 150},
			  		yticklabels={$0$, $100$, $150$},
			  		y label style={at={(axis description cs:0.1,.5)},anchor=south},
			  		every extra y tick/.style={
			  		        tick0/.initial=red,
			  		        tick1/.initial=green,
			  		        tick2/.initial=blue,
			  		        yticklabel style={
			  		            color=\pgfkeysvalueof{/pgfplots/tick\ticknum},},},
			  		extra y ticks ={10,50,120},
			  		extra y tick labels={$10$, $50$, $120$},
			  		legend style={at={(0.97, 0.7)}, anchor=north east},        
			  		legend cell align=left,     
			  		legend entries={$r=10$, $r=50$, $r=120$},
					]
						\addplot[red] table {./5/d/out_r_10_d_0_2.dat};
						\addplot[green] table {./5/d/out_r_50_d_0_2.dat};
						\addplot[blue] table {./5/d/out_r_120_d_0_2.dat};
						\addplot[red, dashed] coordinates {(950,10) (3000,10)};
						\addplot[green, dashed] coordinates {(950,50) (3000,50)};
						\addplot[blue, dashed] coordinates {(950,120) (3000,120)};
						
					\end{axis}
				\end{tikzpicture}				
				\begin{tikzpicture}[scale=0.99]
					%\draw[help lines,step=.2] (0,0) grid (4,4);
					\begin{axis}[
					grid=major,
			  		xmin=950, xmax=1300, 
			  		ymin=-10, ymax=150,     
			  		xlabel={$t \singleSpacing [s]$},
			  		ylabel={$y \singleSpacing [gradi]$},            
			  		title={Uscite del motore per $d=-0.2$}, 
			  		xtick={1000,1100,...,1300},
			  		xticklabels={$1$, $1.1$, $1.2$, $1.3$},
			  		ytick={0, 100, 150},
			  		yticklabels={$0$, $100$, $150$},
			  		y label style={at={(axis description cs:0.1,.5)},anchor=south},
			  		every extra y tick/.style={
			  		        tick0/.initial=red,
			  		        tick1/.initial=green,
			  		        tick2/.initial=blue,
			  		        yticklabel style={
			  		            color=\pgfkeysvalueof{/pgfplots/tick\ticknum},},},
			  		extra y ticks ={10,50,120},
			  		extra y tick labels={$10$, $50$, $120$},
			  		legend style={at={(0.97, 0.7)}, anchor=north east},        
			  		legend cell align=left,     
			  		legend entries={$r=10$, $r=50$, $r=120$},
					]
						\addplot[red] table {./5/d/out_r_10_d__0_2.dat};
						\addplot[green] table {./5/d/out_r_50_d__0_2.dat};
						\addplot[blue] table {./5/d/out_r_120_d__0_2.dat};
						\addplot[red, dashed] coordinates {(950,10) (3000,10)};
						\addplot[green, dashed] coordinates {(950,50) (3000,50)};
						\addplot[blue, dashed] coordinates {(950,120) (3000,120)};
						
					\end{axis}
				\end{tikzpicture}
			\end{figure}
			
\end{document}							