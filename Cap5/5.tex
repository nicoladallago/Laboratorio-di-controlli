\section{Progettazione controllori in spazio di stato}
\label{sec:spazioDiStato}	

	In questa sessione si procede alla realizzazione di due tipi di controllori in spazio di stato.
	
	
		
	\subsection{Controllo in feedforward}
	\label{sub:feedforward}
		
		La rappresentazione in spazio di stato permette di fare una retroazione di stato invece che una retroazione dell'uscita, ed è proprio quello che si fa nel \textit{controllo in feedforward}. 
		
		\begin{figure}[H]
			\centering
			\begin{tikzpicture}[auto, node distance=2cm,>=latex']
				\node [input, name=input] {};
				\node [block, right of=input] (N) {$\bar{N}$};
				\node [sum, right of=N] (sum) {};
				\node [block, right of=sum] (xd) {$\dot{x}=Ax+Bu$};
				\node [block, below of=xd] (u) {$u=Kx$};
				\node [input, name=dirama, right of=xd] {};
				\node [block, right of=dirama] (y) {$y=Cx+Du$};
				\node [output, name=output, right of=y] {};
				
				\draw [->] (input) -- node[pos=0.05] {$r(t)$} (N) {};
				\draw [->] (N) -- node[pos=0.4] {$u_{ext}(t)$} node[pos=0.95] {$+$} (sum) {}; 
				\draw [->] (sum) -- node[pos=0.4] {$u(t)$} (xd) {};
				\draw [-] (xd) -- node[pos=0.5] {$x(t)$} (dirama) {};
				\draw [->] (dirama) |- (u) {};
				\draw [->] (u) -| node[pos=0.95] {$-$} (sum) {};
				\draw [->] (dirama) -- (y) {};
				\draw [->] (y) -- node[pos=0.95] {$y(t)$} (output) {};	
			\end{tikzpicture}
			\caption{Schema a blocchi di un controllo in feedforward.}
			\label{fig:feedforward}
		\end{figure}
		
		\noindent Con una retroazione dallo stato, il controllo, e quindi la matrice $K$, viene scelta piazzando i poli in catena chiusa del sistema. Ricordando che un processo può essere scritto in forma di stato come $P(s)=C(sI-A)^{-1}B$, si nota immediatamente che i poli corrispondono agli autovalori  di $sI-A$. Nel sistema complessivo retroazionato, tali autovalori sono di $sI-A+BK$. Il controllo in feedforward, oltre al piazzamento dei poli permette di modificare anche un parametro scalare $\bar{N}$. Per comprendere lo scopo del parametro $\bar{N}$, si pensi di voler riuscire ad inseguire perfettamente un segnale di ingresso costante:
		
		\begin{align*}
			&r(t)=cost \implies y_{DC}=cost \\
			&y_{DC}=P_{CC}(0)r=-C(A-BK)^{-1}Br
		\end{align*} 
		
		\noindent Per sistemi \textit{SISO} $-C(A-BK)^{-1}B$ si riduce ad uno scalare $c$ per cui $y=cr$ e quindi per poter inseguire il riferimento si usa in blocco $\bar{N}$ che cancella il termine $c$. $\bar{N}$ sarà allora scelto nel seguente modo
		
		\begin{equation}
			\bar{N} = \frac{1}{-C(A-BK)^{-1}B}
			\label{eq:Nbarra}
		\end{equation}
		
		\noindent Un possibile problema di questo tipo di controllo è che se non si conoscono perfettamente i valori nominali delle matrici $(A,B,C,D)$, a regime non si riesce ad inseguire perfettamente il riferimento, perchè
		
		\begin{equation}
			y_{DC}=\bar{N}[-C(A-BK)^{-1}B]r_{DC}=\frac{C(A-BK)^{-1}B}{C_{nom}(A_{nom}-B_{nom}K)^{-1}B_{nom}}r_{DC} \ne r_{DC}
		\end{equation}
		
		\noindent Un altro problema è se è presente un disturbo additivo $d(t)$ in ingresso al controllo, infatti risulterebbe
		
		\begin{equation}
			y_{DC}=\alpha r_{DC} + \beta d_{DC}
		\end{equation} 
		
		
		
		
		
		
		
	\subsection{Controllo integrale}
	\label{subsec:ControlloIntegrale}
	
		\begin{figure}[H]
			\centering
			\begin{tikzpicture}[auto, node distance=1.45cm,>=latex']
				\node [input, name=input] {};
				\node [sum, right of=input] (sum1) {};
				\node [block, right of=sum1] (int) {$\int dt$};
				\node [block, right of=int] (kI) {$K_I$};
				\node [sum, right of=kI] (sum2) {};
				\node [output, name=fittizio3, right of=sum2]{};
				\node [sum, right of=fittizio3] (sum3) {};
				\node [output, name=fittizio0, right of=sum3]{};
				\node [block, right of=fittizio0] (sistema){$\dot x=Ax+Bu$};
				\node [input, name=disturbo, above of=sum3] {};
				\node [block, below of=sum3] (K) {$K$};
				\node [output, name=fittizio, below of=K] {};
				\node [output, name=dirama1, right of=sistema] {};
				\node [block, right of=dirama1] (C){$y=Cx$};
				\node [output, name=dirama2, right of=C] {};
				\node [output, name=output, right of=dirama2] {};
			
				\draw [->] (input) -- node[pos=0.05] {$r(t)$} node[pos=0.95] {$+$}(sum1) {};
				\draw [->] (sum1) -- node[pos=0.5] {$e(t)$} (int) {};
				\draw [->] (int) -- node[pos=0.5] {$-x_I$}(kI) {};
				\draw [->] (kI) -- node[pos=0.95] {$+$} (sum2) {};
				\draw [-] (sum2) --  node[pos=0.95] {$u_{ext}(t)$} (fittizio3) {};
				\draw [->] (fittizio3) -- node[pos=0.95] {$+$}(sum3){};
				\draw [->] (disturbo) -| node[pos=0.05] {$d(t)$} node[pos=0.95] {$+$} (sum3){};
				\draw [-] (sum3) -- node[pos=0.5] {$u(t)$}(fittizio0){};
				\draw [->] (fittizio0) -- (sistema){};
				\draw [-] (sistema) -- (dirama1){};
				\draw [->] (dirama1) -- node[pos=0.3] {$x(t)$} (C){};
				\draw [->] (dirama1) |- (K){};
				\draw [->] (K) -| node[pos=0.95] {$-$}(sum2){};
				\draw [-] (C) -- (dirama2){};	
				\draw [->] (dirama2) -- (output) {};
 				\draw [-] (dirama2) |- (fittizio){};
 				\draw [->] (fittizio) -| node[pos=0.95] {$-$} (sum1){}; 
				\draw [->] (C) -- node[pos=0.95] {$y(t)$}(output){};	
			\end{tikzpicture}
			\caption{Schema a blocchi di un controllo integrale con rumore additivo $d(t)$}
			\label{fig:integrale}
		\end{figure}	
	
		\noindent Il controllo integrale è un altro esempio di controllore che sfrutta la retroazione di stato. Prendiamo in considerazione un sistema scritto in forma di stato, dove per semplicità si considera nulla la matrice $D$ \footnote{del tutto lecito visto che il sistema del motore ha appunto $D=0$}
		
		\begin{equation}
			\begin{cases}
				\dot{x}=Ax+Bu \\
				y=Cx
			\end{cases}
			\label{eq:sistemaNoD}
		\end{equation}
	
		\noindent Al sistema viene aggiunto un nuovo stato
		
		\begin{equation}
			\dot{x}_I=e=y-r=Cx-r
			\label{eq:NuovoStato}
		\end{equation}
	
		\noindent ottenendo ancora un sistema dinamico. Si definisce allora un nuovo \textit{stato aumentato} $z \in \mathbb{R}^{n+1}$
		
		\begin{equation}
			z=
			\begin{bmatrix}
				x_I \\
				x
			\end{bmatrix}
			\label{eq:statoAumentato}
		\end{equation}
	
		\noindent e supposto che l'ingresso sia composto da un ingresso di controllo e un disturbo, come in figura \ref{fig:integrale}, $u=u_{ext}+d$ si ottiene
		
		\begin{equation}
			\begin{cases}
				\dot{z}=A_zz+B_zu_z \\
				y=C_zz
			\end{cases}
			\label{eq:sistemaIntegrale}
		\end{equation}
		
		\noindent dove $u_z = \begin{bmatrix} u_{ext} \\ d \\ r \end{bmatrix} \in \mathbb{R}^3$ e $y_z=y$. Quindi esplicitando l'equazione \ref{eq:sistemaIntegrale} si ricava
		
		\begin{equation}
			\begin{cases}
				\begin{bmatrix}
					\dot{x}_I \\
					x
				\end{bmatrix}
				=
				\underbrace{
				\begin{bmatrix}
					0 & C \\
					0 & A
				\end{bmatrix}
				}_\text{$A_z$}
				\begin{bmatrix}
					x_I \\
					x
				\end{bmatrix}
				+
				\underbrace{
				\begin{bmatrix}
					0 & 0 & -1 \\
					B & B & 0
				\end{bmatrix}
				}_\text{$B_z$}
				\begin{bmatrix}
					u_{ext} \\
					d       \\
					r
				\end{bmatrix} \\
				\\ %
				y=
				\underbrace{
				\begin{bmatrix}
					0 & C
				\end{bmatrix}
				}_\text{$C_z$}
				\begin{bmatrix}
					x_I \\
					x
				\end{bmatrix}
			\end{cases}
			\label{eq:sistemaComplessivo}
		\end{equation}
	
		\noindent Se si definisce $B_z=\begin{bmatrix}B_{ext} & B_{ext} & B_r\end{bmatrix}$ e si sostituisce $u_{ext}=-K_zz$ in \ref{eq:sistemaComplessivo} si ottiene
		
		\begin{equation*}
			\dot{z}=(A_z - B_{ext}K_z)z=
			\begin{bmatrix}
				B_{ext} | B_r
			\end{bmatrix}
			\begin{bmatrix}
				d \\
				r
			\end{bmatrix}
		\end{equation*}
		
		\noindent Quindi, se $A_z-B_{ext}K_z$ è strettamente stabile e gli ingressi sono costanti, $r(t)=r_{DC}$ e $d(t)=d_{DC}$, si rileva che il sistema $z \to z_{cost}$ a $ t \to \infty$.
		\newline Di conseguenza $\dot{z}=\begin{bmatrix}\dot{x}_I \\ \dot{x}\end{bmatrix} \to 0$ che implica $\dot{x}_I \to 0$. Ricordando che $\dot{x}_I=Cx-r=y-r$ si trova che $y \to r$ ottenendo, quindi, un inseguimento perfetto a regime $y \to r \to r_{DC}$.  Se anche $A_z$ e $B_{ext}$ non sono note a priori la matrice $A_z-B_{ext}K_z$ rimane stabile per piccole incertezze, grazie alla \textit{proprietà di continuità} degli autovalori di $A$ rispetto gli elementi di $A$.	
		\newline Rimane ora da verificare che $A_z-B_{ext}K_z$ sia strettamente stabile, questo avviene quando la coppia $(A_z,B_{ext})$ è raggiungibile e quindi con il \textit{criterio PBH} quando $\rank[sI-A_z|B_{ext}]=n+1$. Esplicitando
		
		\begin{equation}
			\begin{bmatrix}
				sI-A_z | B_{ext}
			\end{bmatrix}
			=
			\begin{bmatrix}
			s & -C   & 0 \\
			0 & sI-A & B 
			\end{bmatrix}
		\end{equation}
		
		\noindent si divide la dimostrazione in due parti:
		
		\begin{itemize}
			\item Se $s \ne 0$, $\rank[sI-A_z|B_{ext}]=n+1 \iff \rank[sI-A|B]=n \iff$ la coppia $(A,B)$ è raggiungibile \footnote{questa è una ipotesi necessaria per fare il controllo integrale.}.
			
			\item Se $s=0$, la prima colonna è ininfluente al calcolo e può essere rimossa. Inoltre una eventuale moltiplicazione di righe o colonne per $-1$ così come lo scambio di righe, non cambia il rango, quindi la raggiungibilità per $s=0$ è equivalente a verificare 
			\begin{equation*}
				\rank
				\begin{bmatrix}
					sI-A & -B \\
					C    & 0 
				\end{bmatrix}_{s=0}
				=n+1
			\end{equation*}
			Il rango di questa matrice è $n+1$ se e solo se $s=0$ non è uno zero di $P(s)=C(sI-A)^{-1}B$ \footnote{ulteriore ipotesi necessaria per il controllo integrale.}
		\end{itemize}
		
		\noindent Per quanto riguarda la matrice di retroazione $K$, se le due ipotesi sono verificate, esiste tale che $A_z-B_zK_z$ ha autovalori arbitrari e quindi
		
		\begin{equation}
			u_{ext}(t)=-K_zz(t)=-
			\begin{bmatrix}
				K_I | K
			\end{bmatrix}
			\begin{bmatrix}
				x_I(t) \\
				x(t)
			\end{bmatrix}
			=-K_Ix_I(t)-Kx(t)
		\end{equation}
	
		\noindent Si fa notare che il termine $-K_Ix_I$ può essere scritto come $-K_Ix_I(t)=-K_I\int y(t)-r(t)dt=K_I\int r(t)-y(t)dt$ e rappresentato graficamente in figura \ref{fig:integrale}. 
		\newline  Questo modello risulta più lento rispetto al feed-forward perché l'errore integrale richiede un certo tempo prima che l'effeto sul controllo sia evidente.  
