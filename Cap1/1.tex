\section{Scopo}
\label{sec:Scopo}

	Lo scopo di questa esperienza è la progettazione di regolatori PID e in spazio di stato per il controllo di un motore elettrico a corrente continua controllato in tensione. In particolar modo si vuole analizzare il comportamento del sistema in catena chiusa sollecitato da un ingresso a gradino. Si vogliono quindi caratterizzare le differenze tra un regolatore PID con desaturatore e un regolatore ottenuto tramite retroazione di stato. Altro obiettivo importate dell'esperienza è il confronto tra i risultati ottenuti per simulazione e sperimentalmente, motivando le eventuali ritarature dei parametri di controllo. Separatamente si è voluto anche utilizzare il motore per l'inseguimento di segnali sinusoidali attraverso l'uso del principio del modello interno. 