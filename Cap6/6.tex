\section{Simulazione e esperienza in laboratorio}
\label{sec:simLab}

	In questa sezione si confrontano i dati ottenuti con le simulazioni sul modello del motore e quelli raccolti nell'esperienza di laboratorio, con quanto previsto dalla teoria. In particolare si vogliono testare le differenze prima tra l'uso di un controllore PID con desaturatore, controllo in feed-forward e controllo integrale.
	
	\subsection{Simulazione controllore PID con desaturatore}
	\label{subsec:PIDvsPIDdesa}
	
		Si verificano le prestazioni di un controllo PID con desaturazione in simulazione, utilizzando il modello del motore ricavato nella sezione \ref{subsec:ModellizzazioneMotoriduttore}; in particolare si utilizzano il codice \textit{MATLAB} e i modelli \textit{SIMULINK} riportati rispettivamente nelle appendici \ref{subapp:controllorePID} e \ref{subapp:modelloPID}. 
		\newline Le specifiche da rispettare sono:
		
		\begin{align*}
			t_s &\le 0.30 \singleSpacing [s] \trippleSpacing \textnormal{rispetto a $\pm 1 [gradi]$ del valore a regime} \\
			S &\le 5 \singleSpacing [gradi] \\
			r &= 10,50,120 \singleSpacing [gradi] \\
			d &= \pm 0.5 \singleSpacing [Volt] \\
		\end{align*}
		
		\noindent dove $t_s$ è il tempo di assestamento, $S$ è la sovraelongazione in termini assoluti rispetto al valore di
		riferimento, e $r$ è l'ampiezza del gradino di ingresso. Si utilizza allora la progettazione in frequenza, come nella sezione \ref{sub:ProgettazionePID} per il calcolo dei parametri PID e \ref{subsec:Desaturatore} per il calcolo del parametro di desaturazione. Ovviamente tale metodologia di progettazione prevede diversi gradi di libertà (la scelta di $\alpha \in [\frac{1}{3},\frac{1}{10}]
		$ e $b \in [4, \infty)$) e introduce delle approssimazioni che rendono quindi necessaria ua taratura dei parametri. Infatti scegliendo $\alpha=0.1$ e $b=4$ si ottiene un tempo di salita $t_s=0.022 \singleSpacing [s]$ molto basso ma una sovraelongazione $S=12\%$ per $r=10$ e $d=0.5$, ciò significa che il sistema è troppo veloce. Se invece il disturbo è pari a $d=-0.5$ si ha un tempo di salita $t_s=1.08 \singleSpacing [s]$ troppo lento e una sovraelongazione $S=0.3\%$ sempre con $r=10$.
		
		
		% font e linea grafici MATLAB
		\pgfplotsset{every axis/.append style={
			font=\large,
			line width=1pt,
			tick style={line width=0.8pt}}}
		
		
		\begin{figure}[H]
			\centering
			\begin{tikzpicture}[scale=1, spy using outlines={circle, magnification=5, connect spies}]
				%\draw[help lines,step=.2] (0,0) grid (7,4);
				\begin{axis}[
				grid=major,
		  		xmin=950, xmax=1400, 
		  		ymin=-2, ymax=5.5,     
		  		xlabel={$t \singleSpacing [s]$},
		  		ylabel={$u \singleSpacing [Volt]$},            
		  		title={Ingressi di controllo per $d=0.5$}, 
		  		xtick={900,1000,...,1400},
		  		xticklabels={$0.9$, $1$, $1.1$, $1.2$, $1.3$, $1.4$},
		  		y label style={at={(axis description cs:0.12,.5)},anchor=south},
		  		ytick={-2,-1,...,5},
		  		yticklabels={$-2$, $-1$, $0$, $1$, $2$, $3$, $4$, $5$},
		  		legend pos=north east,         
		  		legend cell align=left,     
		  		legend entries={$r=10$, $r=50$, $r=120$},
				]
					\addplot[red] table {./simulazioni/4/in_r_10_d_0_5.dat};
					\addplot[green] table {./simulazioni/4/in_r_50_d_0_5.dat};
					\addplot[blue] table {./simulazioni/4/in_r_120_d_0_5.dat};
				
					\coordinate (spynode) at (101,680);
					\begin{scope}[fill=white]
						\spy [size=3cm] on (spynode) in node [fill=white] at (6,2);
					\end{scope}
				\end{axis}
			\end{tikzpicture}
			\begin{tikzpicture}[scale=1]
				%\draw[help lines,step=.2] (0,0) grid (4,4);
				\begin{axis}[
				grid=major,
		  		xmin=950, xmax=3000, 
		  		ymin=0, ymax=150,     
		  		xlabel={$t \singleSpacing [s]$},
		  		ylabel={$y \singleSpacing [Volt]$},            
		  		title={Uscite del motore per $d=0.5$}, 
		  		xtick={1000,1500,...,3000},
		  		xticklabels={$1$, $1.5$, $2$, $2.5$, $3$},
		  		ytick={0, 100, 150},
		  		yticklabels={$0$, $100$, $150$},
		  		y label style={at={(axis description cs:0.1,.5)},anchor=south},
		  		every extra y tick/.style={
		  		        tick0/.initial=red,
		  		        tick1/.initial=green,
		  		        tick2/.initial=blue,
		  		        yticklabel style={
		  		            color=\pgfkeysvalueof{/pgfplots/tick\ticknum},},},
		  		extra y ticks ={10,50,120},
		  		extra y tick labels={$10$, $50$, $120$},
		  		legend style={at={(0.97, 0.7)}, anchor=north east},        
		  		legend cell align=left,     
		  		legend entries={$r=10$, $r=50$, $r=120$},
				]
					\addplot[red] table {./simulazioni/4/out_r_10_d_0_5.dat};
					\addplot[green] table {./simulazioni/4/out_r_50_d_0_5.dat};
					\addplot[blue] table {./simulazioni/4/out_r_120_d_0_5.dat};
					\addplot[red, dashed] coordinates {(950,10) (3000,10)};
					\addplot[green, dashed] coordinates {(950,50) (3000,50)};
					\addplot[blue, dashed] coordinates {(950,120) (3000,120)};
					
				\end{axis}
			\end{tikzpicture}	
			\caption{Ingressi di controllo e uscite del motore con controllore PID con desaturatore e rumore additivo $d=0.5$.}
			\label{fig:PIDd_0_5}		
		\end{figure} 
		
		\noindent In figura \ref{fig:PIDd_0_5} sono riportati gli ingressi e le uscite del motore per $d=0.5$, si nota che l'unico ingresso che non satura è quello per $r=10$, mentre gli altri raggiungono la saturazione praticamente subito e per $r=120$ vi rimane abbastanza allungo. Le uscite risultano comunque molto veloci ma con una sovraelongazione troppo elevata.
		
		
		
		
		
		
		
		
		