\section{Simulazione e esperienza in laboratorio}
\label{sec:simLab}

	In questa sezione si confrontano i dati ottenuti con le simulazioni sul modello del motore e quelli raccolti nell'esperienza di laboratorio con quanto previsto dalla teoria. In particolare si vogliono testare le differenze tra l'uso di un controllore PID con desaturatore, controllo in feed-forward e controllo integrale.
	
	\subsection{Simulazione controllore PID con desaturatore}
	\label{subsec:PIDdesaSim}
	
		Si verificano le prestazioni di un controllo PID con desaturazione in simulazione, utilizzando il modello del motore ricavato nella sezione \ref{subsec:ModellizzazioneMotoriduttore}; in particolare si utilizzano il codice \textit{MATLAB} e i modelli \textit{SIMULINK} riportati rispettivamente nelle appendici \ref{subapp:controllorePID} e \ref{subapp:modelloPID}. 
		\newline Le specifiche da rispettare sono:
		
		\begin{align*}
			t_s &\le 0.30 \singleSpacing [s] \trippleSpacing \textnormal{rispetto a $\pm 1 [gradi]$ del valore a regime} \\
			S &\le 5 \singleSpacing [gradi] \\
			r &= 10,50,120 \singleSpacing [gradi] \\
			d &= \pm 0.5 \singleSpacing [Volt] \\
		\end{align*}
		
		\noindent dove $t_s$ è il tempo di assestamento, $S$ è la sovraelongazione in termini assoluti rispetto al valore di
		riferimento e $r$ è l'ampiezza del gradino di ingresso. Si utilizza allora la progettazione in frequenza, come nella sezione \ref{sub:ProgettazionePID} per il calcolo dei parametri PID e \ref{subsec:Desaturatore} per il calcolo del parametro di desaturazione. Ovviamente tale metodologia di progettazione prevede diversi gradi di libertà (la scelta di $\alpha \in [\frac{1}{3},\frac{1}{10}]
		$ e $b \in [4, \infty)$) e introduce delle approssimazioni che possono rendere necessaria una taratura manuale dei parametri. Le simulazioni sono state suddivise per $d=0.5$ e $d=-0.5$, nell'appendice \ref{subapp:PIDsimulazione} si possono osservare tutti i grafici e le tabelle delle simulazioni. Nel primo caso l'uscita del sistema risulta più veloce ma con una sovraelongazione piuttosto marcata, mentre nel secondo è più lenta ma con meno sovraelongazione. Questo si traduce nel fatto che le specifiche non sono rispettate per alcuni segnali di ingresso e si rende necessaria una ritaratura manuale dei parametri. Questa ritaratura però non è del tutto banale, in quanto bisogna diminuire la velocità del sistema per il primo caso e aumentarla nel secondo, si è cercato allora un compromesso. Si è data per prima cosa priorità al tempo di salita per $d=-0.5$ e $r=10$ che superava addirittura il secondo; si è aumentato allora il guadagno integrale per rendere il sistema più veloce. Questo ha però aumentato la sovraelongazione per $d=0.5$ di parecchio, ma per abbassarla si è aumentato anche il guadagno proporzionale, che prima era inferiore all'unità. Alla fine si è dovuto aumentare anche il guadagno proporzionale visto che risultava troppo basso rispetto gli altri due e la sua azione era quasi ininfluente. In conclusione la ritaratura ha portato dei risultati soddisfacenti, riportati nelle figure \ref{fig:PIDd_0_5} e \ref{fig:PIDd__0_5}. Nei grafici si può apprezzare come gli ingressi di controllo saturino quasi subito ma la sovraelongazione rimane comunque bassa grazie all'intervento del desaturatore.
		
		\begin{figure}[H]
			\centering
			\includestandalone[width=1\textwidth]{./simulazioni/4_d_0_5_tuned} 
			\caption{Ingressi di controllo e uscite del motore con controllore PID con desaturatore e rumore additivo $d=0.5$ dopo la taratura manuale.}
			\label{fig:PIDd_0_5}
		\end{figure}
		
		\begin{figure}[H]
			\centering
			\includestandalone[width=1\textwidth]{./simulazioni/4_d__0_5_tuned} 
			\caption{Ingressi di controllo e uscite del motore con controllore PID con desaturatore e rumore additivo $d=-0.5$ dopo la taratura manuale.}
			\label{fig:PIDd__0_5}		
		\end{figure}		
		
		
		
		
	\subsection{Simulazione controllo in feedforward}	
	\label{subsec:feedforwardSim}
	
		In questo caso si vogliono verificare le prestazioni di un controllo in feedforward per rispettare le seguenti specifiche:

		\begin{align*}
			t_s &\le 0.0.15 \singleSpacing [s] \trippleSpacing \textnormal{rispetto a $\pm 5 \%$ del valore a regime} \\
			S &\le 10  \% \\
			r &= 10,50,120 \singleSpacing [gradi] \\
		\end{align*}	
		
		
		
		
		
		