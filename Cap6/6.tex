\section{Simulazione e esperienza in laboratorio}
\label{sec:simLab}

	In questa sezione si confrontano i dati ottenuti con le simulazioni sul modello del motore e quelli raccolti nell'esperienza di laboratorio, con quanto previsto dalla teoria. In particolare si vogliono testare le differenze prima tra l'uso di un controllore PID con desaturatore, controllo in feed-forward e controllo integrale.
	
	\subsection{Simulazione controllore PID con desaturatore}
	\label{subsec:PIDvsPIDdesa}
	
		Si verificano le prestazioni di un controllo PID con desaturazione in simulazione, utilizzando il modello del motore ricavato nella sezione \ref{subsec:ModellizzazioneMotoriduttore}; in particolare si utilizzano il codice \textit{MATLAB} e i modelli \textit{SIMULINK} riportati rispettivamente nelle appendici \ref{subapp:controllorePID} e \ref{subapp:modelloPID}. 
		\newline Le specifiche da rispettare sono:
		
		\begin{align*}
			t_s &\le 0.30 \singleSpacing [s] \trippleSpacing \textnormal{rispetto a $\pm 1 [gradi]$ del valore a regime} \\
			S &\le 5 \singleSpacing [gradi] \\
			r &= 10,50,120 \singleSpacing [gradi] \\
			d &= \pm 0.5 \singleSpacing [Volt] \\
		\end{align*}
		
		\noindent dove $t_s$ è il tempo di assestamento, $S$ è la sovraelongazione in termini assoluti rispetto al valore di
		riferimento e $r$ è l'ampiezza del gradino di ingresso. Si utilizza allora la progettazione in frequenza, come nella sezione \ref{sub:ProgettazionePID} per il calcolo dei parametri PID e \ref{subsec:Desaturatore} per il calcolo del parametro di desaturazione. Ovviamente tale metodologia di progettazione prevede diversi gradi di libertà (la scelta di $\alpha \in [\frac{1}{3},\frac{1}{10}]
		$ e $b \in [4, \infty)$) e introduce delle approssimazioni che possono rendere necessaria una taratura manuale dei parametri. Le simulazioni sono state suddivise per $d=0.5$ e $d=-0.5$. Nel primo caso l'uscita del sistema risulta più veloce ma con una sovraelongazione piuttosto marcata, mentre nel secondo è più lenta ma con meno sovraelongazione.
		
		
		\begin{figure}[H]
			\centering
			\includestandalone[width=1\textwidth]{./simulazioni/4_d_0_5_tuned} 
			\caption{Ingressi di controllo e uscite del motore con controllore PID con desaturatore e rumore additivo $d=0.5$ dopo la taratura manuale.}
			\label{fig:PIDd_0_5}
		\end{figure}
		
		\noindent In entrambi i casi si  nota comunque che gli ingressi di controllo saturano a $5 \singleSpacing [Volt]$ e per $r=120$ rimane saturo per un tempo considerevole rispetto l'evoluzione del sistema.
		

		
		\begin{figure}[H]
			\centering
			\includestandalone[width=1\textwidth]{./simulazioni/4_d__0_5_tuned} 
			\caption{Ingressi di controllo e uscite del motore con controllore PID con desaturatore e rumore additivo $d=-0.5$ dopo la taratura manuale.}
			\label{fig:PIDd__0_5}		
		\end{figure}		
		
		
		
		
		
		
		